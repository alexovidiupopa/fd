%! Author = alexp
%! Date = 5/4/2025

% Preamble
\documentclass[11pt]{article}

% Packages
\usepackage{amsmath}
\usepackage{listings}
\usepackage{xcolor}
\usepackage{url}
\usepackage{graphicx}

\lstdefinelanguage{JavaScript}{
    keywords={typeof, new, true, false, catch, function, return, null, catch, switch, var, if, in, while, do, else, case, break},
    keywordstyle=\color{blue}\bfseries,
    ndkeywords={class, export, boolean, throw, implements, import, this},
    ndkeywordstyle=\color{darkgray}\bfseries,
    identifierstyle=\color{black},
    sensitive=false,
    comment=[l]{//},
    morecomment=[s]{/*}{*/},
    commentstyle=\color{purple}\ttfamily,
    stringstyle=\color{red}\ttfamily,
    morestring=[b]',
    morestring=[b]"
}

% Code block styling
\lstset{
    language=JavaScript,
    basicstyle=\ttfamily\small,
    keywordstyle=\color{blue},
    commentstyle=\color{green},
    stringstyle=\color{red},
    breaklines=true,
    numbers=left,
    numberstyle=\tiny\color{gray},
    frame=single,
}

% Document
\begin{document}

    \title{GeoMap Component Documentation}
    \author{Alex-Ovidiu POPA}
    \date{\today}
    \maketitle

    \section{Introduction}
    The \texttt{GeoMap} component (Figure \ref{fig:enter-label}) is a React-based interactive map built using the Leaflet library and its React integration (\texttt{react-leaflet}). It provides functionalities such as marker management (adding, editing, deleting), marker clustering, and filtering by name. This document elaborates on each functionality and includes the corresponding code snippets.
    \begin{figure}
        \centering
        \includegraphics[width=1\linewidth]{component.png}
        \caption{Embedded view of the GeoMap component}
        \label{fig:enter-label}
    \end{figure}

    \section{Libraries Used}
    The following libraries are used in the \texttt{GeoMap} component:
    \begin{itemize}
        \item \textbf{Leaflet}: A JavaScript library for interactive maps. (\url{https://leafletjs.com/})
        \item \textbf{react-leaflet}: React bindings for Leaflet. (\url{https://react-leaflet.js.org/})
        \item \textbf{react-leaflet-markercluster}: A plugin for clustering markers on the map. (\url{https://github.com/yuzhva/react-leaflet-markercluster})
        \item \textbf{@changey/react-leaflet-markercluster}: A modern fork of the marker cluster plugin.
    \end{itemize}

    \section{Main Functionalities}
    The \texttt{GeoMap} component provides the following features:

    \subsection{Marker Management}
    \textbf{Description:} Users can add, edit, and delete markers on the map. Markers are stored in the component's state and can be persisted using \texttt{localStorage}.

    \textbf{Add Marker:}
    Markers can be added by clicking on the map in "Add Marker Mode." The following function handles adding a new marker:
    \begin{lstlisting}[language=javascript]
const addMarker = () => {
if (newMarkerPosition && newMarkerName.trim()) {
    setMarkers([...markers, { position: newMarkerPosition, name: newMarkerName }]);
    setNewMarkerPosition(null);
    setNewMarkerName('');
    setIsAddMarkerMode(false);
}
};
    \end{lstlisting}

    Moreover, a custom function component is needed in order to correctly handle on-map click events, as React's Leaflet does not detect changes at the level of the DOM:
    \begin{lstlisting}[language=javascript]
const MapClickHandler: React.FC = () => {
        useMapEvent('click', (e) => {
            if (isAddMarkerMode) {
                setNewMarkerPosition([e.latlng.lat, e.latlng.lng]);
            }
        });
        return null;
    };
    \end{lstlisting}

    \textbf{Edit Marker:}
    Markers can be edited by selecting them and modifying their name or position:
    \begin{lstlisting}[language=javascript]
const saveEditedMarker = () => {
if (editingMarker !== null && editedPosition) {
    const updatedMarkers = [...markers];
    updatedMarkers[editingMarker] = {
        name: editedName,
        position: editedPosition,
    };
    setMarkers(updatedMarkers);
    setEditingMarker(null);
    setEditedName('');
    setEditedPosition(null);
}
};
    \end{lstlisting}

    \textbf{Delete Marker:}
    Markers can be removed from the map using the following function:
    \begin{lstlisting}[language=javascript]
const deleteMarker = (index: number) => {
const updatedMarkers = markers.filter((_, i) => i !== index);
setMarkers(updatedMarkers);
};
    \end{lstlisting}

    \subsection{Marker Clustering}
    \textbf{Description:} Markers are automatically grouped into clusters for better visualization when there are many markers on the map. This is achieved using the \texttt{react-leaflet-markercluster} library.

    \textbf{Implementation:}
    \begin{lstlisting}[language=javascript]
<MarkerClusterGroup>
{filteredMarkers.map((marker, index) => (
    <Marker key={index} position={marker.position}>
        <Popup>
            <p>{marker.name}</p>
        </Popup>
    </Marker>
))}
</MarkerClusterGroup>
    \end{lstlisting}

    \subsection{Search Functionality}
    \textbf{Description:} Users can filter markers by name using a search input. The search query is matched against the marker names.

    \textbf{Implementation:}
    \begin{lstlisting}[language=javascript]
const filteredMarkers = markers.filter((marker) =>
marker.name.toLowerCase().includes(searchQuery.toLowerCase())
);
    \end{lstlisting}

    \subsection{Mode Switching}
    \textbf{Description:} The map can toggle between "Add Marker Mode" (for adding markers) and "Move Mode" (for navigating the map). In "Add Marker Mode," dragging is disabled.

    \textbf{Implementation:}
    \begin{lstlisting}[language=javascript]
const MapModeHandler: React.FC<{ isAddMarkerMode: boolean }> = ({ isAddMarkerMode }) => {
const map = useMap();

React.useEffect(() => {
    if (isAddMarkerMode) {
        map.dragging.disable();
    } else {
        map.dragging.enable();
    }
}, [isAddMarkerMode, map]);

return null;
};
    \end{lstlisting}

    \subsection{Persistent Storage}
    \textbf{Description:} Markers can be saved to and loaded from \texttt{localStorage}, ensuring persistence across sessions. Although a trivial approach, this may be replaced in real-world applications with an API call, if desired.

    \textbf{Save Markers:}
    \begin{lstlisting}[language=javascript]
const saveMarkers = () => {
localStorage.setItem('markers', JSON.stringify(markers));
alert('Markers saved successfully!');
};
    \end{lstlisting}

    \textbf{Load Markers:}
    \begin{lstlisting}[language=javascript]
useEffect(() => {
const savedMarkers = localStorage.getItem('markers');
if (savedMarkers) {
    setMarkers(JSON.parse(savedMarkers));
}
}, []);
    \end{lstlisting}

    \section{Component Integration}

    In order to integrate the component within a React application, one must simply import it, and use it as shown below:

    \begin{lstlisting}[language=javascript]
import GeoMap from './components/GeoMap';

const App: React.FC = () => {
    return (
        <div>
            <h1>Geographical Data Map Component</h1>
            <GeoMap />
        </div>
    );
};
    \end{lstlisting}
\end{document}